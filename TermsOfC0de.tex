\documentclass[12pt, unicode, a4paper]{ltjsreport}

\usepackage[top=25truemm, bottom=25truemm, left=20truemm, right=20truemm]{geometry}
\usepackage[unicode, hidelinks, bookmarksnumbered, pdfusetitle]{hyperref}
\usepackage{chngcntr}
\usepackage{tocloft}

\counterwithout{section}{chapter}
\renewcommand{\thesection}{第\arabic{section}条}

\cftsetindents{chapter}{0truemm}{20truemm}
\cftsetindents{section}{5truemm}{20truemm}

\title{C0de 部内規約}
\date{令和4年(2024年)1月nn日}

\begin{document}

\maketitle

\tableofcontents

\chapter{総則}
    \section{名称}
        \begin{enumerate}
            \item 当部は名古屋工業大学公認課外活動団体であり、「C0de」と称す。
        \end{enumerate}

    \section{設立日}
        \begin{enumerate}
            \item 当部の設立日は平成28年(2016年)4月1日である。
        \end{enumerate}

    \section{所在地}
        \begin{enumerate}
            \item 当部の所在地は 〒466-8555 名古屋市昭和区御器所町名古屋工業大学55号館2階第1共用室 である。ただし、郵便その他配達物については 同大学19号館学生センター横の郵便ポスト にて受け付ける。
        \end{enumerate}

    \section{目的}\label{purpose}
        \begin{enumerate}
            \item 当部は「co-develop」(仲間を集め、ある対象について開発すること)を目的とする。
        \end{enumerate}


\chapter{入部・退部}
    \section{入部}
        \begin{enumerate}
            \item 当部は以下のすべての要件を満たした者による入部を受け付ける。
            \begin{itemize}
                \item 名古屋工業大学または名古屋工業大学大学院に在学中である。
                \item 当規約\ref{budget}2項によって議決された金額を部費として支払った。
                \item 以下のすべての情報を部に提出した。
                \begin{itemize}
                    \item 氏名
                    \item 学籍番号
                    \item 現住所
                    \item メールアドレス
                \end{itemize}
                \item 当規約に同意した。
            \end{itemize}
        \end{enumerate}

    \section{退部}
        \begin{enumerate}
            \item 役員を除くすべての部員は部長に退部の意思を表明することで任意の時期に退部することができる。
            \item 役員は当規約\ref{resignation}の規定に基づき役員を辞任した場合、前項の規定に基づき退部することができる。
            \item 部員が死亡したときは退部したものとみなす。
        \end{enumerate}


\chapter{懲戒}
    \section{懲戒}\label{disciplinary}
        \begin{enumerate}
            \item 部員の過半数以上が出席する部会においてその出席者の3分の2以上の承認が得られた場合、当部は部員に対して以下の懲戒処分を下すことができる。ただし、部員資格剥奪(強制退部)を行う際は役員全員の承認を必要とする。
            \begin{itemize}
                \item 厳重注意:部長より厳重に注意を行う。
                \item 戒告:部会にて顛末を説明し、役員より厳重注意を行う。
                \item 部員資格停止:部員としての資格を一時的に停止する。停止期間中は部に関わる設備を利用できない。
                \item 部員資格剥奪(強制退部):懲戒対象の意志に関わらず部員の資格を喪失する。
            \end{itemize}
            \item 懲戒事由は以下の通りとする。
            \begin{itemize}
                \item 公序良俗・法令等に違反し、反社会的な行動を行った。
                \item 正当な理由なく部費を支払わなかった。
                \item 上記のほか、部の秩序を著しく乱すようなことを行った。
            \end{itemize}
            \item 懲戒審議に際して、懲戒対象となり得る部員は部会または役員会にて釈明を行うことができる。釈明に対して部員は十分な配慮と酌量を行わなければならない。
            \item 懲戒を行うことが議決されてから実際に懲戒を行うまでには、1週間以上の猶予期間を設けなければならない。
            \item 部長は懲戒対象となった部員に対し、懲戒を行う旨を猶予期間中に合理的な手段で通達しなければならない。
            \item 通達なしでの懲戒は無効とする。ただし、通達に懲戒対象の部員が応答しない場合、これは通達したものとみなす。
        \end{enumerate}


\chapter{役員}
    \section{役員}\label{directors}
        \begin{enumerate}
            \item 当部は部の円滑な運営のために役員をおくものとする。各役員が務める役職とその定員は以下の通りとする。
            \begin{itemize}
                \item 部長(1名):当部の代表
                \item 副部長(1名):部長の補佐
                \item 会計(1名):部費の管理
                \item 庶務(2名程度):行事の運営
            \end{itemize}
            \item 役員の任期は当規約\ref{activity}1項にて定められた活動期間とする。
            \item 部会にて当規約\ref{decision}に基づいた承認を得ることで、新たな役職をおくこと、役員の定員を変更することができる。
            \item 各役員は他の役職を兼任することができる。
        \end{enumerate}

    \section{役員の選出}\label{election}
        \begin{enumerate}
            \item 次年の役員の選出は当規約\ref{activity}1項にて定められた活動期間の最後の月の部会において、部員の互選によって行う。
            \item 前項にて規定された互選によって役員が決定しない場合は、前年度の部長が適任だと考える者を指名する。
            \item 前項の規定により指名を受けた部員はやむを得ない状況を除いて指名を受諾しなければならない。
            \item 役員の再選は妨げないものとする。
        \end{enumerate}

    \section{役員の辞任・解任}\label{resignation}
        \begin{enumerate}
            \item 部長は、役員会で辞任の意思を表明し、部員の過半数以上が出席する部会においてその出席者の3分の2以上の承認が得られた場合、部長を辞任することができる。
            \item 部長以外の役員は、役員会で辞任の意思を表明し、部会にて当規約\ref{decision}に基づいた承認が得られた場合、辞任することができる。
            \item 当規約\ref{disciplinary}に違反するなど役員として不適格と認められた者は、部員の過半数以上が出席する部会においてその出席者の3分の2以上の承認が得られた場合、強制的に解任させられる。
            \item 前項の規定により役員が解任される場合、後任の役員は当規約\ref{election}1項にて規定される互選時期に関わらず、直ちに当規約\ref{election}に基づき選定される。
        \end{enumerate}

    \section{役員会}
        \begin{enumerate}
            \item 役員会は任意の役員が必要に応じ他の役員を召集することによって不定期に開催される。
            \item 役員会における承認・議決は、役員の懲戒・解任審議を除いて原則全会一致とする。
            \item 必要に応じて、役員以外の部員・在学OB/OG・OB/OGを役員会に参加させることができる。ただし、その者の議決権は原則ないものとする。
        \end{enumerate}


\chapter{OB/OG}
    \section{在学OB/OG}
        \begin{enumerate}
            \item 名古屋工業大学第4学年または名古屋工業大学大学院に在籍する学生で、当部に部員として所属していたものを在学OB/OGとする。ただし、退部したものを除く。
            \item 在学OB/OGとなる権利を有する者が引き続き部員として当部に所属することを希望する場合、役員会における承認をもって前項の規定に関わらず引き続き部員として当部に所属することを認める。その場合、その者には部費の負担・役員への選出・部会での議決権が原則発生するものとする。
        \end{enumerate}

    \section{OB/OG}
        \begin{enumerate}
            \item 名古屋工業大学、名古屋工業大学大学院のどちらにも在籍しないもののうち、当部に部員として所属していたものをOB/OGとする。ただし、退部したものを除く。
        \end{enumerate}


\chapter{活動}
    \section{活動}\label{activity}
        \begin{enumerate}
            \item 当部の活動期間は毎年1月1日より12月31日までとする。
            \item 当部は当規約\ref{purpose}の目的を達するために以下の活動を行う。
            \begin{itemize}
                \item プロジェクト
                \item チュートリアル
                \item 当規約\ref{meeting}によって規定される部会
                \item 当規約\ref{event}によって規定される行事
                \item 新入生の勧誘活動
            \end{itemize}
            \item 全ての部員は当規約\ref{purpose}の目的を達するため、前項の活動に積極的に参加することが求められるが、学業を優先させることは妨げられない。
        \end{enumerate}

    \section{行事}\label{event}
        \begin{enumerate}
            \item 当部は行事として以下のものを開催する。
            \begin{itemize}
                \item 新歓BBQ
                \item 夏合宿
                \item 忘年会
            \end{itemize}
        \end{enumerate}


\chapter{部会}
    \section{部会}\label{meeting}
        \begin{enumerate}
            \item 部会は以下のことを行うために開催される。
            \begin{itemize}
                \item 当部の運営のために必要な審議・議決
                \item プロジェクトの活動報告
                \item 連絡事項の通達
            \end{itemize}
            \item 部会は原則毎月第2水曜日に開催されるものとする。なお、開始時刻・開催場所は部会ごとに役員が決定する。
            \item 役員は必要に応じて臨時部会を開催することができる。
        \end{enumerate}

    \section{部会における審議・議決}\label{decision}
        \begin{enumerate}
            \item {\bf 部会の議決は他に定めのない限り、部員の3分の1以上が出席する部会において、その出席者の過半数以上の賛成によって可決され、過半数以上の反対によって否決されるものとする。ただし、賛成・反対が同数の場合は役員会における議決を部会の議決とする。}
            \item 懲戒審議と役員の解任審議は他のどのような審議よりも優先される。
            \item 部会における審議の際の発言の責任は、著しく部の秩序を乱すもので無い限り問われないものとする。
            \item 部会における議決の際は、当規約とは別に定める書式に基づく議決書を原則作成するものとする。
        \end{enumerate}


\chapter{部費}
    \section{部費}\label{budget}
        \begin{enumerate}
            \item 部費は以下の物品を購入するために使用される。
            \begin{itemize}
                \item デバッグ用の端末
                \item 参考書
                \item 当部のサーバのパーツ
                \item 当部が活動する上で必要となる日用品
            \end{itemize}
            \item 部費の金額は当規約\ref{activity}1項にて定められた活動期間の最初の月の部会までに役員会において審議・議決されるとする。
            \item 部費に損害が生じた場合は原則役員が責任を負う。ただし、損害が著しい場合はこの限りではない。
        \end{enumerate}

    \section{部費の管理・監査}
        \begin{enumerate}
            \item 部費の管理は当規約\ref{directors}にて規定される会計が行う。
            \item 部費の管理を円滑に行うために当部は以下の銀行口座を保有する。
            \begin{itemize}
                \item ゆうちょ銀行 記号:12030 番号:15378031
            \end{itemize}
            \item 部費の監査は会計以外の役員が行う。
        \end{enumerate}

    \section{部費の徴収・返金}
        \begin{enumerate}
            \item 新入部員以外の部員は当規約\ref{budget}2項によって議決された金額を、部費として当規約\ref{activity}1項にて規定された活動期間の最初の月中に当部の銀行口座へ振り込まなければならない。
            \item 新入部員は当規約\ref{budget}2項によって議決された金額を、部費として入部後1ヶ月以内に当部の銀行口座へ振り込まなければならない。
            \item 追加の部費を徴収する場合は、部員の過半数以上が出席する部会において、会計がその時点までの会計報告を行った上で、その出席者の3分の2以上の承認を得なければならない。
            \item 会計は当規約\ref{activity}1項にて定められた活動期間の最後の月の部会にて会計報告を行う。残金は次年の部費に積み立てる。
            \item 退部時の部費の返却は一切行わないものとする。ただし、部会にて当規約\ref{decision}の規定に基づいた承認が得られた場合はその限りではない。
        \end{enumerate}

    \section{部費による物品やサービスの購入}
        \begin{enumerate}
            \item 部費によって物品やサービスを購入する際は、事前または事後に役員会における承認を必要とする。
            \item 部費によって物品やサービスを購入する際に立替が必要となる場合、その立替は任意の部員が行う。
            \item 会計は前項の規定に基づき立替を行った部員に対し、可能な限り早く無利子で返金するものとする。
        \end{enumerate}


\chapter{改定}
    \section{改定}
        \begin{enumerate}
            \item 当規約は部員の過半数以上が出席する部会においてその出席者の3分の2以上の承認が得られた場合、改定することができる。ただし、当該部会を開催する1週間以上前にSlack等の第三者サービスで改定箇所を示さなければならない。
            \item 改定規則は原則即時施行とする。ただし、部会の議決によって公布期間を設けることができる。
        \end{enumerate}


\chapter{免責}
    \section{免責}
        \begin{enumerate}
            \item 当部は重過失が認められない限り、当部の活動に伴い生じた事故や事件等に関していかなる責任も負わず、各個人の責任とする。ただし、部会にて当規約\ref{decision}に基づいた承認が得られた場合はこの限りではない。
        \end{enumerate}
    \section{保険}
        \begin{enumerate}
            \item 部員が当規約\ref{activity}に規定される活動を行うときは不慮の事故等に備え、損害賠償保険や人身傷害保険などの保険に加入しなければならない。
            \item 保険に加入しない部員に対して役員はその部員の活動を制限することができる。
        \end{enumerate}

    \section{個人情報・データの取扱い}
        \begin{enumerate}
            \item 当部の個人情報の取り扱いについては、原則「個人情報の保護に関する法律」第1条の目的を達することを目標とし、厳重にこれを管理する。
            \item 書面または電磁的記録媒体で保存されているデータは可能な範囲でバックアップを行う。
            \item サーバ管理者は重過失が認められない限り、電磁的記録媒体で保存されているデータが喪失したとしても一切の責任を負わない。
            \item 部員が退部した場合は退部した部員に関わる個人情報・データは原則速やかに破棄される。ただし、当該部員との合意が得られた場合はこの限りではない。
        \end{enumerate}


\chapter{その他}
    \section*{その他}
        当規約に定めるほかに必要な事項は十分に配慮の上、役員会・部会でこれを別に定める。

    \section*{発行}
        当規約は、発行日である 平成29年(2017年)4月1日から発効する。
        \begin{itemize}
            \item 平成29年(2017年)4月1日:当規約の発行
            \item 平成29年(2017年)8月9日:当規約の大幅な改定
            \item 平成29年(2017年)12月31日:部会と部門会の大幅な改定
            \item 平成31年(2019年)2月13日:当規約の大幅な改定
            \item 令和元年(2019年)11月13日:当規約の大幅な改定
            \item 令和2年(2020年)5月13日:所在地、入部審査、部会の出席義務、部費についての改定
            \item 令和4年(2022年)3月9日:所在地についての改定
            \item 令和5年(2023年)1月11日:役員についての改定
            \item 令和6年(2024年)1月nn日:部費の残金返却の廃止
        \end{itemize}

    \section*{代表者による証明}
        当規約の記載内容について、事実と相違ないことをここに証明する。
        \begin{itemize}
            \item 代表者
            \begin{flushright}
                (印)
            \end{flushright}
            \item 住所
        \end{itemize}

\end{document}
